% Copyright (c) 2009-2014 by the University of Waikato, Hamilton, NZ. 
% This work is made available under the terms of the 
% Creative Commons Attribution-ShareAlike 3.0 license, 
% http://creativecommons.org/licenses/by-sa/3.0/. 
%
% Version: $Revision: 3363 $

\documentclass[a4paper]{book}

\usepackage{wrapfig}
\usepackage{graphicx}
\usepackage{hyperref}
\usepackage{multirow}
\usepackage{scalefnt}
\usepackage{tikz}
\usepackage{varwidth}

% watermark -- for draft stage
\usepackage[firstpage]{draftwatermark}
\SetWatermarkLightness{0.9}
\SetWatermarkScale{5}

\hyphenation{ImageMagick}
\hyphenation{ImageJ}

% Version: $Revision$

\newenvironment{tight_itemize}{
\begin{itemize}
  \setlength{\itemsep}{1pt}
  \setlength{\parskip}{0pt}
  \setlength{\parsep}{0pt}}{\end{itemize}
}

\newenvironment{tight_enumerate}{
\begin{enumerate}
  \setlength{\itemsep}{1pt}
  \setlength{\parskip}{0pt}
  \setlength{\parsep}{0pt}}{\end{enumerate}
}

% if you just need a simple heading
% Usage:
%   \heading{the text of the heading}
\newcommand{\heading}[1]{
  \vspace{0.3cm} \noindent \textbf{#1} \newline
}

\newcommand{\icon}[1]{\tikz[baseline=-3pt]\node[inner sep=0pt,outer sep=0pt]{\includegraphics[height=1.1em]{#1}};}


\title{
  \textbf{ADAMS} \\
  {\Large \textbf{A}dvanced \textbf{D}ata mining \textbf{A}nd \textbf{M}achine
  learning \textbf{S}ystem} \\
  {\Large Module: adams-imaging} \\
  \vspace{1cm}
  \includegraphics[width=2cm]{images/imaging-module.png} \\
}
\author{
  Peter Reutemann
}

\setcounter{secnumdepth}{3}
\setcounter{tocdepth}{3}

\begin{document}

\begin{titlepage}
\maketitle

\thispagestyle{empty}
\center
\begin{table}[b]
	\begin{tabular}{c l l}
		\parbox[c][2cm]{2cm}{\copyright 2009-2014} &
		\parbox[c][2cm]{5cm}{\includegraphics[width=5cm]{images/coat_of_arms.pdf}} \\
	\end{tabular}
	\includegraphics[width=12cm]{images/cc.png} \\
\end{table}

\end{titlepage}

\tableofcontents
\listoffigures
%\listoftables


%%%%%%%%%%%%%%%%%%%%%%%%%%%%%%%%%%%
\chapter{ADAMS}
ADAMS has custom image processing support that does not rely on other libraries.

The following actors are available:
\begin{tight_itemize}
	\item \texttt{sink.ImageWriter} -- writes an image container to a file
	using the specified writer.
	\item \texttt{transformer.BufferedImageTransformer} -- performs a transformation
	using an existing transformer class on the incoming image and
	outputs another image again.
	\item \texttt{transformer.BufferedImageFeatureGenerator} -- turns a
	\texttt{BufferedImageContainer} into an \texttt{weka.core.Instance} object to
	be used in WEKA. The attaced meta-data in form of a report can be added to the
	output object as well.
	\item \texttt{transformer.ImageReader} -- reads an image file using the 
	specified image reader.
\end{tight_itemize}

Figure \ref{adams-blur-flow} shows a
flow\footnote{adams-imaging-gaussian\_blur.flow} for reading images, blurring
them using a gaussian blur transformer and displaying them side-by-side. Figures
\ref{adams-blur-output-original} and \ref{adams-blur-output-blurred} show original
and blurred image.

\begin{figure}[htb]
  \centering
  \includegraphics[width=10.0cm]{images/adams-blur-flow.png}
  \caption{Flow for blurring images stored in a directory.}
  \label{adams-blur-flow}
\end{figure}

\begin{figure}[htb]
  \begin{minipage}[b]{0.48\linewidth}
  \centering
  \includegraphics[height=3.7cm]{images/adams-blur-output-original.png}
  \caption{The original image.}
  \label{adams-blur-output-original}
  \end{minipage}%
  \begin{minipage}[b]{0.48\linewidth}
  \centering
  \includegraphics[height=3.7cm]{images/adams-blur-output-blurred.png}
  \caption{The blurred image.}
  \label{adams-blur-output-blurred}
  \end{minipage}
\end{figure}



%%%%%%%%%%%%%%%%%%%%%%%%%%%%%%%%%%%
\chapter[Java Advanced
Imaging]{\parbox[b]{1.2cm}{\includegraphics[width=1cm]{images/jai.png}}\parbox[c]{14cm}{Java
Advanced Imaging}}
Java Advanced Imaging (JAI) is an API to provide a simple, high-level
programming model which allows developers to create their own image manipulation
routines\footnote{\url{http://en.wikipedia.org/wiki/Java_Advanced_Imaging}{}}.

The following actors are available:
\begin{tight_itemize}
    \item \texttt{source.JAICreateImage} -- for creating an empty image to be 
    forwarded.
\end{tight_itemize}

Reading and writing images are done using the \textit{ImageReader} transformer
and \textit{ImageWriter} sink:
\begin{tight_itemize}
	\item \texttt{ImageReader} -- use the \textit{JAIImageReader}
	\item \texttt{ImageWriter} -- use the \textit{JAIImageWriter}
\end{tight_itemize}

Since the JAI actors, readers and writers use \texttt{BufferedImageContainer}, the 
\texttt{BufferedImageTransformer} and \texttt{BufferedImageFeatureGenerator}
transformers can be used.


%%%%%%%%%%%%%%%%%%%%%%%%%%%%%%%%%%%
\chapter[ImageJ]{\parbox[b]{1.2cm}{\includegraphics[width=1cm]{images/imagej.png}}\parbox[c]{14cm}{ImageJ}}
ImageJ is a public domain software suite written in Java (using AWT, opposed to
Swing which ADAMS uses) for image processing, developed at National Institutes 
of Health (\cite{imagej}).

\section{Flow}
There are four ImageJ actors available:
\begin{tight_itemize}
	\item \texttt{transformer.ImageJReader} -- for reading any image file that
	JAI supports\footnote{\url{http://imagejdocu.tudor.lu/doku.php?id=faq:general:which_file_formats_are_supported_by_imagej}{}}
	and forwarding an \texttt{ImagePlusContainer} object.
	\item \texttt{transformer.ImageJTransformer} -- performs a transformation
	using an existing ImageJ transformer class on the incoming image and
	outputs another image again. ImageJ plugin filters, commands and pre-recorded
	macros can be used to perform transformations.
	\item \texttt{transformer.ImageJFeatureGenerator} -- turns an
	\texttt{ImagePlusContainer} into an \texttt{weka.core.Instance} object to
	be used in WEKA. The attaced meta-data in form of a report can be added to the
	output object as well.
	\item \texttt{transformer.ImageJReleaseAllImages} -- removes all images
	currently listed in ImageJ's batch mode list, freeing up memory.
	\item \texttt{sink.ImageJReleaseImage} -- removes the incoming image from
	ImageJ's batch mode list of images, freeing up memory.
	\item \texttt{sink.ImageJWriter} -- for writing an \texttt{ImagePlusContainer}
	to a file format that ImageJ supports. If the image type cannot be
	determined based on the extension, you can also specify which type to generate.
\end{tight_itemize}

Figure \ref{imagej-greyscale-flow} shows a
flow\footnote{adams-imaging-transform\_to\_greyscale.flow} for reading images,
turning them into greyscale using a  transformer and displaying them
side-by-side.
Figures \ref{imagej-greyscale-output-original} and
\ref{imagej-greyscale-output-grey} show original and greyscale image.

\begin{figure}[htb]
  \centering
  \includegraphics[width=10.0cm]{images/imagej-greyscale-flow.png}
  \caption{ImageJ flow for turning images stored in a directory into greyscale
  ones.}
  \label{imagej-greyscale-flow}
\end{figure}

\begin{figure}[htb]
  \begin{minipage}[b]{0.48\linewidth}
  \centering
  \includegraphics[height=3.7cm]{images/imagej-greyscale-output-original.png}
  \caption{The original image.}
  \label{imagej-greyscale-output-original}
  \end{minipage}%
  \begin{minipage}[b]{0.48\linewidth}
  \centering
  \includegraphics[height=3.7cm]{images/imagej-greyscale-output-grey.png}
  \caption{The greyscale image.}
  \label{imagej-greyscale-output-grey}
  \end{minipage}
\end{figure}

\clearpage
\section{Plugins}
By default, ADAMS includes plugins located in the following 
directory on Linux/Unix/Mac:
\begin{verbatim}
  $HOME/.adams/imagej/plugins
\end{verbatim}
and on Windows here:
\begin{verbatim}
  %USERPROFILE%/_adams/imagej/plugins
\end{verbatim}
You can override this directory by using the \texttt{ADAMS\_IMAGEJ\_DIR}
environment variable, which defines the directory one level above the 
\textit{plugins} directory. For instance, if your plugins directory is located
at:
\begin{verbatim}
  /home/user/imagej/plugins
\end{verbatim}
You have to define the \texttt{ADAMS\_IMAGEJ\_DIR} environment variable as 
follows:
\begin{verbatim}
  ADAMS_IMAGEJ_DIR=/home/user/imagej
\end{verbatim}


%%%%%%%%%%%%%%%%%%%%%%%%%%%%%%%%%%%
\chapter[ImageMagick]{\parbox[b]{1.2cm}{\includegraphics[width=1cm]{images/imagemagick.png}}\parbox[c]{14cm}{ImageMagick}}
ImageMagick$\textsuperscript{\textregistered}$ is a software suite to create,
edit, compose, or convert bitmap images (\cite{imagemagick}). On Windows, in order to
process images with ImageMagick, you need to set the \texttt{IM\_TOOLPATH} 
environment variable, pointing to the installation. Similar, for dcraw, you 
need to defined the \texttt{DCRAW\_TOOLPATH} variable and, for ufraw, the
\texttt{UFRAW\_TOOLPATH} one.

There are three ImageMagick actors available:
\begin{tight_itemize}
	\item \texttt{transformer.ImageMagickOperation} -- performs the ImageMagick
	(convert, dcraw, ufraw) operations that the user selected from the class
	hierarchy.
	\item \texttt{transformer.ImageMagickTransformer} -- performs any ImageMagick
	command on the incoming image that the \texttt{convert} tool\footnote{\url{http://www.imagemagick.org/script/convert.php}{}} supports and
	outputs another image again.
\end{tight_itemize}

Reading and writing images are done using the \textit{ImageReader} transformer
and \textit{ImageWriter} sink:
\begin{tight_itemize}
	\item \texttt{ImageReader} -- use the \textit{ImageMagickImageReader} or \textit{UIfrawImageReader}
	\item \texttt{ImageWriter} -- use the \textit{ImageMagickImageWriter}
\end{tight_itemize}

There is no separate transformer for generating a WEKA instance, since the
ImageMagick actors process and output \texttt{BufferedImageContainer} objects as
well, just like the JAI actors. You can use the \texttt{BufferedImageFeatureGenerator} for
generating WEKA output.

The example flow\footnote{adams-imaging-imagemagick\_script.flow} in Figure
\ref{imagemagick-resize-flow} loads a single photo from disk and then uses
ImageMagick to resize it to 90 by 90 pixels and scaling it by 200\% (see
\ref{imagemagick-resize-script}). Finally, the modified image is displayed in
the image viewer.

\begin{figure}[htb]
  \centering
  \includegraphics[width=10.0cm]{images/imagemagick-resize-flow.png}
  \caption{ImageMagick flow for processing (resizing) a single image.}
  \label{imagemagick-resize-flow}
\end{figure}

\begin{figure}[htb]
  \begin{center}
  \begin{varwidth}{\textwidth}
\begin{verbatim}
# resizing image
-resize 90x90
# scaling it up again
-scale 200%
\end{verbatim}
  \end{varwidth}
  \end{center}
  \caption{ImageMagick commands to resizing.}
  \label{imagemagick-resize-script}
\end{figure}

\begin{figure}[htb]
  \begin{minipage}[b]{0.48\linewidth}
  \centering
  \includegraphics[height=3.8cm]{images/imagemagick-resize-original.png}
  \caption{The original image.}
  \label{imagemagick-resize-original}
  \end{minipage}%
  \begin{minipage}[b]{0.48\linewidth}
  \centering
  \includegraphics[height=3.8cm]{images/imagemagick-resize-output.png}
  \caption{The resized image.}
  \label{imagemagick-resize-output}
  \end{minipage}
\end{figure}

% %%%%%%%%%%%%%%%%%%%%%%%%%%%%%%%%%%
\chapter[BoofCV]{\parbox[b]{1.2cm}{\includegraphics[width=1cm]{images/boofcv.png}}\parbox[c]{14cm}{BoofCV}}
BoofCV is an API for real-time computer vision and robotics applications\footnote{\url{http://boofcv.org/}{}}.

There are two BoofCV actors available:
\begin{tight_itemize}
	\item \texttt{transformer.BoofCVTransformer} -- performs a transformation
	using an existing BoofCV transformer class on the incoming image and
	outputs another image again.
	\item \texttt{transformer.BoofCVDetectLines} -- detects lines in images
	using a Hough line detector based on polar parametrization.
	\item \texttt{transformer.BoofCVFeatureGenerator} -- turns a
	\texttt{BoofCVImageContainer} into an \texttt{weka.core.Instance} object to
	be used in WEKA. The attaced meta-data in form of a report can be added to the
	output object as well.
\end{tight_itemize}


%%%%%%%%%%%%%%%%%%%%%%%%%%%%%%%%%%%
\chapter{LIRE}
The Lucene Image Retrieval library \cite{lire} provides a wide range of feature
generators that work on \textit{BufferedImageContainer} objects.


%%%%%%%%%%%%%%%%%%%%%%%%%%%%%%%%%%%
\chapter{Object conversion}
JAI and ImageMagick actors generate and accept a different type of token,
\textit{BufferedImageContainer} namely, which cannot be processed by ImageJ
actors. Vice versa, the tokens generated by ImageJ actors, of type
\textit{ImagePlusContainer}, are not accepted by JAI/ImagMagick actors. In order
to exchange data between the two domains, the \textit{Convert} transformer can
once again be used. 

The following conversions are available to convert from one
format into another:
\begin{tight_itemize}
	\item \textit{BoofCVImageToBufferedImage} -- for BoofCV to JAI/ImageMagick
	conversion.
	\item \textit{BufferedImageToBoofCV} -- for JAI/ImageMagick to BoofCV
	conversion.
	\item \textit{BufferedImageToImageJ} -- for JAI/ImageMagick to ImageJ
	conversion.
	\item \textit{ColorToHex} -- turns a Color object into its hexa-decimal 
	notation.
	\item \textit{ImageJToBufferedImage} -- converting from ImageJ to
	JAI/ImageMagick.
	\item \textit{HexToColor} -- turns a color in hexa-decimal notation back 
	into a Color object.
\end{tight_itemize}


%%%%%%%%%%%%%%%%%%%%%%%%%%%%%%%%%%%
\chapter{OCR}
A common task in image processing is \textit{optical character recognition} 
(OCR). ADAMS offers a simple wrapper around the open-source \textit{tesseract} 
engine \cite{tesseract}. The engine is available for Windows, Linux and Mac OSX.
It supports multiple languages, however, these need to be installed in order to
be actually available.

The follwoing actors are available:
\begin{tight_itemize}
	\item \textit{TesseractConfiguration} -- standalone for configuring OCR, 
	mainly to define where the tesseract executable is located.
	\item \textit{TesseractOCR} -- this transformer turns an image file into
	one or more text files, which need to be further processed in the flow
	then.\footnote{adams-imaging-ocr.flow}
\end{tight_itemize}

By default, the \textit{TesseractConfiguration} standalone uses the globally
defined preferences as default values. In the preferences dialog 
(\textit{Main menu $\rightarrow$ Program $\rightarrow$ Preferences 
$\rightarrow$ Tesseract}) you can specify the location of the tesseract
executable and the default language (see Figure \ref{tesseract-preferences}).
\begin{figure}[htb]
  \centering
  \includegraphics[width=10.0cm]{images/tesseract-preferences.png}
  \caption{Preferences for tesseract.}
  \label{tesseract-preferences}
\end{figure}


%%%%%%%%%%%%%%%%%%%%%%%%%%%%%%%%%%%
\chapter{Interaction}
The \textit{PixelSelector} transformer allows the user to interact with the
flow. The interaction with the user works as follows: an image viewer instance
is displayed when the \textit{PixelSelector} transformer receives an image token
as input. The use then right-clicks on a pixel that he wants to process, e.g.,
labelling for WEKA data generation. After all the pixels have been selected and
processed, the user then hits the \textit{OK} button to close the dialog. The
\textit{PixelSelector} then forwards the image container with the
attached, enriched report for further processing.

The \textit{PixelSelector} transformer is very generic, which means the actor
is responsible for the actions that the user can select from the right-click
menu. This is done by selecting the appropriate actions from the list of
available ones, e.g., \textit{AddClassification} (package
\texttt{adams.flow.transformer.pixelselector}), which is used for attaching
classification labels to pixels. In order to make these selections visible not
just in the report that is displayed on the right-hand side in the dialog,
appropriate overlays can be selected as well, e.g., the
\textit{ClassificationOverlay} (package
\texttt{adams.flow.transformer.pixelselector}) overlay, which displays the
pixels with the associated labels on the screen.

Figure \ref{pixelselector-flow} shows a
flow\footnote{adams-imaging-pixelselector.flow} that lets the user hand-label
all JPG images in a directory and generated WEKA data from it. It uses a cropped
region of 5x5 pixels around the selected pixels for the data generation. The
user interface for selecting the pixels is shown in Figure
\ref{pixelselector-interaction} and a resulting dataset in Figure
\ref{pixelselector-dataset}.

\begin{figure}[htb]
  \centering
  \includegraphics[width=10.0cm]{images/pixelselector-flow.png}
  \caption{Flow for generating ARFF file from user-labelled pixels.}
  \label{pixelselector-flow}
\end{figure}

\begin{figure}[htb]
  \centering
  \includegraphics[width=10.0cm]{images/pixelselector-interaction.png}
  \caption{User interface for labelling pixels, displaying some pixels
  labelled already.}
  \label{pixelselector-interaction}
\end{figure}

\begin{figure}[htb]
  \centering
  \includegraphics[width=10.0cm]{images/pixelselector-dataset.png}
  \caption{Example dataset generated using the PixelSelector.}
  \label{pixelselector-dataset}
\end{figure}

Of course, due to the interactive nature, labelling is performed on-the-fly and
no record is kept. Once the image has been processed, the
\textit{PixelSelector} will forget about it. If you want to preserve the
attached report, you can use the \textit{ReportFileWriter} transformer to save
the report to disk. 

In order to re-use a previously saved report, you can use the
\textit{SetReportFromFile} or \textit{SetReportFromSource} transformer to
replace the default report in the image container after you loaded the image
with the one stored on disk. This allows you to continue work with previously
generated labels, saving you a lot of work.

Since the \textit{SetReportFromFile} and \textit{SetReportFromSource}
transformers generate \textit{ReportHandler} tokens, you need to explicitly
cast the type of the tokens to the desired one, e.g.,
\textit{BufferedImageContainer}, using the \textit{Cast} control actor.

% %%%%%%%%%%%%%%%%%%%%%%%%%%%%%%%%%%
\chapter{Feature output}
Of course, the data can be turned into a format that is suitable for machine 
learning applications like WEKA (\cite{weka}). For JAI and ImageMagick 
transformers, both generating \textit{BufferedImageContainer} tokens, the 
\textit{BufferedImageFeatureGenerator} can be used to
generate such output. For ImageJ generated tokens, outputting 
\textit{ImagePlusContainer} tokens, you have to use the 
\textit{ImageJFeatureGenerator} instead. What kind of output is generated,
depends on the \textit{feature converter} defined in those feature generator
transformers. By default, spreadsheet data is generated, which can be stored
in CSV files. Figure \ref{imagej-csv-generation-flow} shows a
flow\footnote{adams-imaging-csv\_generation.flow} that generates a CSV file
from images using ImageJ. The resulting dataset, as displayed in the spreadsheet
viewer, is shown in Figure \ref{imagej-csv-generation-dataset}.

\begin{figure}[htb]
  \centering
  \includegraphics[width=10.0cm]{images/imagej-csv-generation-flow.png}
  \caption{Generating a CSV file using ImageJ.}
  \label{imagej-csv-generation-flow}
\end{figure}

\begin{figure}[htb]
  \centering
  \includegraphics[width=10.0cm]{images/imagej-csv-generation-dataset.png}
  \caption{The ImageJ generated CSV file.}
  \label{imagej-csv-generation-dataset}
\end{figure}


%%%%%%%%%%%%%%%%%%%%%%%%%%%%%%%%%%%
\chapter{Miscellaneous actors}
The imaging module offers some more actors that have not been introduced yet.

\noindent Available sources:
\begin{tight_itemize}
	\item \textit{ColorProvider} -- outputs Color objects generated by a 
	configured color provider.
\end{tight_itemize}

\noindent Available transformers:
\begin{tight_itemize}
	\item \textit{Draw} -- Performs draw operations on images, like setting 
	pixels, drawing lines, rectangles, ovals, text, images\footnote{adams-imaging-draw.flow}.
	\item \textit{ImageInfo} -- Allows you to obtain \textit{width} and
	\textit{height} information from an image.
	\item \textit{ImageMetaData} -- Extracts meta-data (EXIF or IPTC) from an
	image as spreadsheet using the Sanselan library\cite{sanselan}.\footnote{adams-imaging-meta\_data.flow}
	\item \textit{ImageMetaDataExtractor} -- Extracts meta-data (EXIF or IPTC) from an
	image as spreadsheet using the Meta-Data Extractor library\cite{metadataextractor}.
	\item \textit{LocateObjects} -- provides a framework for algorithms that
	locate objects in images.
\end{tight_itemize}

\noindent Available sinks:
\begin{tight_itemize}
  \item \textit{FFmpeg} -- actor for processing videos using
  ffmpeg\cite{ffmpeg}\footnote{adams-imaging-ffmpeg.flow}.
\end{tight_itemize}


%%%%%%%%%%%%%%%%%%%%%%%%%%%%%%%%%%%
% This work is made available under the terms of the
% Creative Commons Attribution-ShareAlike 4.0 license,
% http://creativecommons.org/licenses/by-sa/4.0/.

\begin{thebibliography}{999}
	% to make the bibliography appear in the TOC
	\addcontentsline{toc}{chapter}{Bibliography}

    % references
	\bibitem{adams}
		\textit{ADAMS} -- Advanced Data mining and Machine learning System \\
		\url{https://adams.cms.waikato.ac.nz/}{}
		
	\bibitem{poi}
		\textit{Apache POI} -- the Java API for Microsoft Documents \\
		\url{http://poi.apache.org/}{}

	\bibitem{fastexcel}
		\textit{fastexcel} -- Generate and read big Excel files quickly \\
		\url{https://github.com/dhatim/fastexcel}{}

\end{thebibliography}


\end{document}
