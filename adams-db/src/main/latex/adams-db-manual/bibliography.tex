% Copyright (c) 2009-2018 by the University of Waikato, Hamilton, NZ.
% This work is made available under the terms of the 
% Creative Commons Attribution-ShareAlike 4.0 license,
% http://creativecommons.org/licenses/by-sa/4.0/.

\begin{thebibliography}{999}
	% to make the bibliography appear in the TOC
	\addcontentsline{toc}{chapter}{Bibliography}

    % references
	\bibitem{adams}
		\textit{ADAMS} -- Advanced Data mining and Machine learning System \\
		\url{https://adams.cms.waikato.ac.nz/}{}
		
	\bibitem{sqlite}
		\textit{SQLite} -- SQLite is a C-language library that implements
		a small, fast, self-contained, high-reliability, full-featured,
		SQL database engine. \\
		\url{https://sqlite.org/}{}

	\bibitem{sqlitexerial}
		\textit{SQLite JDBC Driver} -- SQLite JDBC, developed by
		Taro L. Saito, is a library for accessing and creating SQLite
		database files in Java.  \\
		\url{https://github.com/xerial/sqlite-jdbc}{}

	\bibitem{hsqldb}
		\textit{HSQLDB} -- HSQLDB (HyperSQL Database) is a relational
		database engine written in Java. \\
		\url{http://hsqldb.org/}{}

	\bibitem{postgresql}
		\textit{PostgreSQL} --  is a powerful, open source object-relational
		database system with over 30 years of active development that has
		earned it a strong reputation for reliability, feature robustness,
		and performance.  \\
		\url{https://www.postgresql.org/}{}

	\bibitem{jtds}
		\textit{jTDS} -- is an open source 100\% pure
		Java (type 4) JDBC 3.0 driver for Microsoft SQL Server (6.5, 7,
		2000, 2005, 2008 and 2012) and Sybase Adaptive Server Enterprise
		(10, 11, 12 and 15). \\
		\url{http://jtds.sourceforge.net/}{}

	\bibitem{mssql}
		\textit{MS SQL Server} -- SQL database engine developed by Microsoft \\
		\url{https://www.microsoft.com/sql-server}{}

	\bibitem{mssqljdbc}
		\textit{JDBC driver for MS SQL Server} -- The Microsoft JDBC Driver
		for SQL Server is a Type 4 Java Database Connectivity (JDBC) 4.2
		compliant driver that provides robust data access to SQL Server 2017,
		SQL Server 2016, SQL Server 2014, SQL Server 2012, SQL Server 2008 R2,
		SQL Server 2008, and Azure SQL Database. \\
		\url{https://docs.microsoft.com/en-us/sql/connect/jdbc/overview-of-the-jdbc-driver}{}

\end{thebibliography}
