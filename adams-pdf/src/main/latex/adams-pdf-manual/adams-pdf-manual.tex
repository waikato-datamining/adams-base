% Copyright (c) 2012-2013 by the University of Waikato, Hamilton, NZ. 
% This work is made available under the terms of the 
% Creative Commons Attribution-ShareAlike 4.0 license,
% http://creativecommons.org/licenses/by-sa/4.0/.
%
% Version: $Revision$

\documentclass[a4paper]{book}

\usepackage{wrapfig}
\usepackage{graphicx}
\usepackage{hyperref}
\usepackage{multirow}
\usepackage{scalefnt}
\usepackage{tikz}

% watermark -- for draft stage
\usepackage[firstpage]{draftwatermark}
\SetWatermarkLightness{0.9}
\SetWatermarkScale{5}

% Version: $Revision$

\newenvironment{tight_itemize}{
\begin{itemize}
  \setlength{\itemsep}{1pt}
  \setlength{\parskip}{0pt}
  \setlength{\parsep}{0pt}}{\end{itemize}
}

\newenvironment{tight_enumerate}{
\begin{enumerate}
  \setlength{\itemsep}{1pt}
  \setlength{\parskip}{0pt}
  \setlength{\parsep}{0pt}}{\end{enumerate}
}

% if you just need a simple heading
% Usage:
%   \heading{the text of the heading}
\newcommand{\heading}[1]{
  \vspace{0.3cm} \noindent \textbf{#1} \newline
}

\newcommand{\icon}[1]{\tikz[baseline=-3pt]\node[inner sep=0pt,outer sep=0pt]{\includegraphics[height=1.1em]{#1}};}


\title{
  \textbf{ADAMS} \\
  {\Large \textbf{A}dvanced \textbf{D}ata mining \textbf{A}nd \textbf{M}achine
  learning \textbf{S}ystem} \\
  {\Large Module: adams-pdf} \\
  \vspace{1cm}
  \includegraphics[width=2cm]{images/pdf-module.png} \\
}
\author{
  Peter Reutemann
}

\setcounter{secnumdepth}{3}
\setcounter{tocdepth}{3}

\begin{document}

\begin{titlepage}
\maketitle

\thispagestyle{empty}
\center
\begin{table}[b]
	\begin{tabular}{c l l}
		\parbox[c][2cm]{2cm}{\copyright 2012-2016} &
		\parbox[c][2cm]{5cm}{\includegraphics[width=5cm]{images/coat_of_arms.pdf}} \\
	\end{tabular}
	\includegraphics[width=12cm]{images/cc.png} \\
\end{table}

\end{titlepage}

\tableofcontents
\listoffigures
%\listoftables

%%%%%%%%%%%%%%%%%%%%%%%%%%%%%%%%%%%
\chapter{Introduction}
The \textit{pdf} module adds PDF authoring capabilities to ADAMS. 
This is possible thanks to the iText \cite{itext} and jPod Renderer \cite{jpod} 
libraries for creating, manipulating and viewing of PDF files.

%%%%%%%%%%%%%%%%%%%%%%%%%%%%%%%%%%%
\chapter{Flow}
The following source actors are available:
\begin{tight_itemize}
	\item \textit{PDFNewDocument} -- creates an empty PDF document.\footnote{adams-pdf-create\_pdf2.flow}
\end{tight_itemize}
The following transformers are available:
\begin{tight_itemize}
	\item \textit{PDFAppendDocument} -- appends a PDF document.\footnote{adams-pdf-create\_pdf2.flow}
	\item \textit{PDFCreate} -- for creating PDFs, using text files,
	spreadsheets and images.\footnote{adams-pdf-create\_pdf.flow}
	\item \textit{PDFExtract} -- extracts a range of pages.
	\item \textit{PDFExtractImages} -- extracts the images from a PDF.\footnote{adams-pdf-extract\_images.flow}
	\item \textit{PDFExtractText} -- obtaining the plain text of the PDF.\footnote{adams-pdf-extract\_text.flow}
	\item \textit{PDFMerge} -- merges several PDFs into a single one.\footnote{adams-pdf-page\_count.flow}
	\item \textit{PDFPageCount} -- determines the number of pages in a
	PDF.\footnote{adams-pdf-page\_count.flow}
	\item \textit{PDFRenderPages} -- renders pages of a PDF as images.\footnote{adams-pdf-render\_pages.flow}
	\item \textit{PDFStamp} -- allows to add an overlay to a PDF.\footnote{adams-pdf-page\_overlay.flow}
\end{tight_itemize}
The following sinks are available:
\begin{tight_itemize}
	\item \textit{PDFCloseDocument} -- closes a PDF document, writes out the content
	to disk.\footnote{adams-pdf-create\_pdf2.flow}
	\item \textit{PDFViewer} -- for viewing PDFs.\footnote{adams-pdf-view\_pdf.flow}
\end{tight_itemize}

\heading{Creating PDFs}
Generating a PDF using the \textit{PDFCreate} transformer is really easy.
The transformer takes an array of file names as input, which will all get added
to the specified output PDF file. Basic options, like page size and orientation,
can be set as well. How and what files get added to the PDF, is determined by
the ``proclets'', i.e., little processor classes, that you specify and configure:
\begin{tight_itemize}
	\item \textit{SpreadSheet} -- for adding spreadsheet files as tables.
	\item \textit{Headline} -- for adding a headline.
	\item \textit{Image} -- for adding images (GIF, JPEG, PNG).
	\item \textit{PageBreak} -- forces a pagebreak.
	\item \textit{PlainText} -- for adding plain text files as paragraphs.
	\item \textit{Rectangle} -- draws a rectangle at specified location.
\end{tight_itemize}

Figure \ref{pdf-create-flow} shows a flow\footnote{adams-pdf-create\_pdf.flow} 
that adds all files (i.e., three) found in a directory to a single PDF. 
Figures \ref{pdf-create-output1}, \ref{pdf-create-output2} and 
\ref{pdf-create-output3} show the resulting pages of the PDF in the viewer.

\begin{figure}[htb]
  \centering
  \includegraphics[width=10.0cm]{images/pdf-create-flow.png}
  \caption{Flow for creating a PDF file from various sources.}
  \label{pdf-create-flow}
\end{figure}

\begin{figure}[htb]
  \centering
  \includegraphics[width=10.0cm]{images/pdf-create-output1.png}
  \caption{CSV files get added as tables.}
  \label{pdf-create-output1}
\end{figure}

\begin{figure}[htb]
  \centering
  \includegraphics[width=10.0cm]{images/pdf-create-output2.png}
  \caption{Images can get inserted as well.}
  \label{pdf-create-output2}
\end{figure}

\begin{figure}[htb]
  \centering
  \includegraphics[width=10.0cm]{images/pdf-create-output3.png}
  \caption{Plain text files get added as simple text.}
  \label{pdf-create-output3}
\end{figure}

%%%%%%%%%%%%%%%%%%%%%%%%%%%%%%%%%%%
\chapter{Tools}
The \textit{PDF viewer} can be used to load and browse PDF files. Figure 
\ref{pdf-viewer} shows a screenshot of the viewer with a PDF file loaded.

\begin{figure}[htb]
  \centering
  \includegraphics[width=10.0cm]{images/pdf-viewer.png}
  \caption{Viewer for PDF files.}
  \label{pdf-viewer}
\end{figure}

%%%%%%%%%%%%%%%%%%%%%%%%%%%%%%%%%%%
\chapter{Troubleshooting}
\begin{tight_itemize}
	\item \textbf{Problem:} 64bit Linux shows error message when viewing PDF 
	that `libfreetype.so' library was not found (``UnsatisfiedLinkError"). \\
	\textbf{Solution:} Make sure that /usr/lib/libfreetype.so exists. If not, 
	add a symbolic link to the 64bit library, using a similar command as 
	follows (for Kubuntu 11.10 or LinuxMint 11, 13):
\begin{verbatim}
  sudo ln -s /usr/lib/x86_64-linux-gnu/libfreetype.so.6 /usr/lib/libfreetype.so
\end{verbatim}
    And restart the application.
\end{tight_itemize}

%%%%%%%%%%%%%%%%%%%%%%%%%%%%%%%%%%%
% This work is made available under the terms of the
% Creative Commons Attribution-ShareAlike 4.0 license,
% http://creativecommons.org/licenses/by-sa/4.0/.

\begin{thebibliography}{999}
	% to make the bibliography appear in the TOC
	\addcontentsline{toc}{chapter}{Bibliography}

    % references
	\bibitem{adams}
		\textit{ADAMS} -- Advanced Data mining and Machine learning System \\
		\url{https://adams.cms.waikato.ac.nz/}{}
		
	\bibitem{poi}
		\textit{Apache POI} -- the Java API for Microsoft Documents \\
		\url{http://poi.apache.org/}{}

	\bibitem{fastexcel}
		\textit{fastexcel} -- Generate and read big Excel files quickly \\
		\url{https://github.com/dhatim/fastexcel}{}

\end{thebibliography}


\end{document}
