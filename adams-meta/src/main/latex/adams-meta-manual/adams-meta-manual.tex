% Copyright (c) 2009-2014 by the University of Waikato, Hamilton, NZ. 
% This work is made available under the terms of the 
% Creative Commons Attribution-ShareAlike 4.0 license,
% http://creativecommons.org/licenses/by-sa/4.0/.
%
% Version: $Revision$

\documentclass[a4paper]{book}

\usepackage{wrapfig}
\usepackage{graphicx}
\usepackage{hyperref}
\usepackage{multirow}
\usepackage{scalefnt}
\usepackage{tikz}
\usepackage{varwidth}

% watermark -- for draft stage
\usepackage[firstpage]{draftwatermark}
\SetWatermarkLightness{0.9}
\SetWatermarkScale{5}

% Version: $Revision$

\newenvironment{tight_itemize}{
\begin{itemize}
  \setlength{\itemsep}{1pt}
  \setlength{\parskip}{0pt}
  \setlength{\parsep}{0pt}}{\end{itemize}
}

\newenvironment{tight_enumerate}{
\begin{enumerate}
  \setlength{\itemsep}{1pt}
  \setlength{\parskip}{0pt}
  \setlength{\parsep}{0pt}}{\end{enumerate}
}

% if you just need a simple heading
% Usage:
%   \heading{the text of the heading}
\newcommand{\heading}[1]{
  \vspace{0.3cm} \noindent \textbf{#1} \newline
}

\newcommand{\icon}[1]{\tikz[baseline=-3pt]\node[inner sep=0pt,outer sep=0pt]{\includegraphics[height=1.1em]{#1}};}


\title{
  \textbf{ADAMS} \\
  {\Large \textbf{A}dvanced \textbf{D}ata mining \textbf{A}nd \textbf{M}achine
  learning \textbf{S}ystem} \\
  {\Large Module: adams-meta} \\
  \vspace{1cm}
  \includegraphics[width=2cm]{images/meta-module.png} \\
}
\author{
  Peter Reutemann
}

\setcounter{secnumdepth}{3}
\setcounter{tocdepth}{3}

\begin{document}

\begin{titlepage}
\maketitle

\thispagestyle{empty}
\center
\begin{table}[b]
	\begin{tabular}{c l l}
		\parbox[c][2cm]{2cm}{\copyright 2009-2014} &
		\parbox[c][2cm]{5cm}{\includegraphics[width=5cm]{images/coat_of_arms.pdf}} \\
	\end{tabular}
	\includegraphics[width=12cm]{images/cc.png} \\
\end{table}

\end{titlepage}

\tableofcontents
\listoffigures
%\listoftables


%%%%%%%%%%%%%%%%%%%%%%%%%%%%%%%%%%%

\newpage
\chapter{Dynamic use of templates}
\label{dynamic}
The templating mechanism described in the ``core-module'' manual, shows how to 
speed up the inception of new flows. But the templates can also be used in a
dynamic way at runtime using the following actors:
\begin{tight_itemize}
	\item \textit{TemplateStandalone} -- for templates that generate standalones
	\item \textit{TemplateSource} -- for templates that generate sources
	\item \textit{TemplateTransformer} -- for templates that generate transforming
	sub-flows
	\item \textit{TemplateSink} -- for templates that generate sinks
\end{tight_itemize}
The sub-flow generation is done in a lazy way, i.e., only when the
aforementioned template actor is executed, the template is generated. The
sub-flow is used till either the end of the flow execution or if a variable
changes that is attached to the template itself. In the latter case, the
sub-flow gets re-generated the next time the template actor gets executed. This
dynamic sub-flow generation in conjunction with variable use, allows to adapt
and change the flow at runtime. The example \textit{adams-core-template.flow}
demonstrates this.

\newpage
\chapter{Copying callable actors}
\label{copycallableactors}
Callable actors can not only be used as synchronization points in the flow. It
is also possible to \textit{copy} them, using them as templates. If you don't
want to use external flows, but still need to use the same sub-flow multiple
times and avoid the bottle next of synchronous execution, then you can use
one of the following actors to create a copy of the callable at the very same 
location:
\begin{tight_itemize}
	\item \textit{CopyCallableStandalone} -- copies a callable standalone
	\item \textit{CopyCallableSource} -- copies a callable source
	\item \textit{CopyCallableTransformer} -- copies a callable transformer
	\item \textit{CopyCallableSink} -- copies a callable sink
\end{tight_itemize}

\newpage
\chapter{Including external actors}
\label{includeexteranlactors}
Similar to the external actors of the \textit{adams-core} module, the following
actors allow the use of external flow snippets. However, these flows simply
replace themselves with the content of the external flow and cannot be changed
dynamically with variables. Flexibility has been traded here for performance.
\begin{tight_itemize}
	\item \textit{IncludeExternalStandalone} -- includes an external standalone
	\item \textit{IncludeExternalSource} -- includes an external source
	\item \textit{IncludeExternalTransformer} -- includes an external transformer
	\item \textit{IncludeExternalSink} -- includes an external sink
\end{tight_itemize}


%%%%%%%%%%%%%%%%%%%%%%%%%%%%%%%%%%%
% This work is made available under the terms of the
% Creative Commons Attribution-ShareAlike 4.0 license,
% http://creativecommons.org/licenses/by-sa/4.0/.

\begin{thebibliography}{999}
	% to make the bibliography appear in the TOC
	\addcontentsline{toc}{chapter}{Bibliography}

    % references
	\bibitem{adams}
		\textit{ADAMS} -- Advanced Data mining and Machine learning System \\
		\url{https://adams.cms.waikato.ac.nz/}{}
		
	\bibitem{poi}
		\textit{Apache POI} -- the Java API for Microsoft Documents \\
		\url{http://poi.apache.org/}{}

	\bibitem{fastexcel}
		\textit{fastexcel} -- Generate and read big Excel files quickly \\
		\url{https://github.com/dhatim/fastexcel}{}

\end{thebibliography}


\end{document}
