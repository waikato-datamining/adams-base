% Copyright (c) 2014 by the University of Waikato, Hamilton, NZ. 
% This work is made available under the terms of the 
% Creative Commons Attribution-ShareAlike 4.0 license,
% http://creativecommons.org/licenses/by-sa/4.0/.
%
% Version: $Revision$

\documentclass[a4paper]{book}

\usepackage{wrapfig}
\usepackage{graphicx}
\usepackage{hyperref}
\usepackage{multirow}
\usepackage{scalefnt}
\usepackage{tikz}

% watermark -- for draft stage
\usepackage[firstpage]{draftwatermark}
\SetWatermarkLightness{0.9}
\SetWatermarkScale{5}

% Version: $Revision$

\newenvironment{tight_itemize}{
\begin{itemize}
  \setlength{\itemsep}{1pt}
  \setlength{\parskip}{0pt}
  \setlength{\parsep}{0pt}}{\end{itemize}
}

\newenvironment{tight_enumerate}{
\begin{enumerate}
  \setlength{\itemsep}{1pt}
  \setlength{\parskip}{0pt}
  \setlength{\parsep}{0pt}}{\end{enumerate}
}

% if you just need a simple heading
% Usage:
%   \heading{the text of the heading}
\newcommand{\heading}[1]{
  \vspace{0.3cm} \noindent \textbf{#1} \newline
}

\newcommand{\icon}[1]{\tikz[baseline=-3pt]\node[inner sep=0pt,outer sep=0pt]{\includegraphics[height=1.1em]{#1}};}


\title{
  \textbf{ADAMS} \\
  {\Large \textbf{A}dvanced \textbf{D}ata mining \textbf{A}nd \textbf{M}achine
  learning \textbf{S}ystem} \\
  {\Large Module: adams-osm} \\
  \vspace{1cm}
  \includegraphics[width=2cm]{images/osm-module.png} \\
}
\author{
  Peter Reutemann
}

\setcounter{secnumdepth}{3}
\setcounter{tocdepth}{3}

\begin{document}

\begin{titlepage}
\maketitle

\thispagestyle{empty}
\center
\begin{table}[b]
	\begin{tabular}{c l l}
		\parbox[c][2cm]{2cm}{\copyright 2014} &
		\parbox[c][2cm]{5cm}{\includegraphics[width=5cm]{images/coat_of_arms.pdf}} \\
	\end{tabular}
	\includegraphics[width=12cm]{images/cc.png} \\
\end{table}

\end{titlepage}

\tableofcontents
\listoffigures
%\listoftables

%%%%%%%%%%%%%%%%%%%%%%%%%%%%%%%%%%%
\chapter{Introduction}
\textit{OpenStreetMap}\cite{osm} is built by a community of mappers that contribute and
maintain data about roads, trails, cafés, railway stations, and much more, all
over the world.

\begin{figure}[htb]
  \centering
  \includegraphics[width=6.0cm]{images/osm_logo.png}
  \label{osm_logo}
\end{figure}

The \textit{JMapViewer} component \cite{jmapviewer} displays the image tiles 
returned by a Mapnik server (either the official OpenStreetMap one or a 
user-defined one, e.g., an in-house one) in Java.

%%%%%%%%%%%%%%%%%%%%%%%%%%%%%%%%%%%
\chapter{Flow}
The following sinks are available:
\begin{tight_itemize}
	\item \textit{OpenStreetMapViewer} -- displays a map with the option
	of displaying custom markers, circles, rectangles and polygons using layers.
\end{tight_itemize}
The following conversions are available:
\begin{tight_itemize}
	\item \textit{SpreadSheetToMapObjects} -- can use various generators for 
	creating map objects (markers, circles, rectangles, polygons) with 
	optional meta-data to be displayed in layers in a 
	\textit{OpenStreetMapViewer} sink.
\end{tight_itemize}

%%%%%%%%%%%%%%%%%%%%%%%%%%%%%%%%%%%
\chapter{Tools}
The \textit{OpenStreetMap viewer} is a simple tool for viewing maps using
the official OpenStreetMap site (internet access required). 
Figure \ref{openstreetmapviewer} shows a map centered around Hamilton, NZ.
The viewer allows you to center the map around a provided GPS location,
using the decimal notation for coordinates (``lat [,] lon'').

\begin{figure}[htb]
  \centering
  \includegraphics[width=10.0cm]{images/openstreetmapviewer.png}
  \caption{OpenStreetMap viewer showing Hamilton, NZ.}
  \label{openstreetmapviewer}
\end{figure}

%%%%%%%%%%%%%%%%%%%%%%%%%%%%%%%%%%%
% This work is made available under the terms of the
% Creative Commons Attribution-ShareAlike 4.0 license,
% http://creativecommons.org/licenses/by-sa/4.0/.

\begin{thebibliography}{999}
	% to make the bibliography appear in the TOC
	\addcontentsline{toc}{chapter}{Bibliography}

    % references
	\bibitem{adams}
		\textit{ADAMS} -- Advanced Data mining and Machine learning System \\
		\url{https://adams.cms.waikato.ac.nz/}{}
		
	\bibitem{poi}
		\textit{Apache POI} -- the Java API for Microsoft Documents \\
		\url{http://poi.apache.org/}{}

	\bibitem{fastexcel}
		\textit{fastexcel} -- Generate and read big Excel files quickly \\
		\url{https://github.com/dhatim/fastexcel}{}

\end{thebibliography}


\end{document}
