% This work is made available under the terms of the
% Creative Commons Attribution-ShareAlike 3.0 license, 
% http://creativecommons.org/licenses/by-sa/3.0/. 
%
% Version: $Revision$

\documentclass[a4paper]{book}

\usepackage{wrapfig}
\usepackage{graphicx}
\usepackage{hyperref}
\usepackage{multirow}
\usepackage{scalefnt}
\usepackage{tikz}

% watermark -- for draft stage
\usepackage[firstpage]{draftwatermark}
\SetWatermarkLightness{0.9}
\SetWatermarkScale{5}

% Version: $Revision$

\newenvironment{tight_itemize}{
\begin{itemize}
  \setlength{\itemsep}{1pt}
  \setlength{\parskip}{0pt}
  \setlength{\parsep}{0pt}}{\end{itemize}
}

\newenvironment{tight_enumerate}{
\begin{enumerate}
  \setlength{\itemsep}{1pt}
  \setlength{\parskip}{0pt}
  \setlength{\parsep}{0pt}}{\end{enumerate}
}

% if you just need a simple heading
% Usage:
%   \heading{the text of the heading}
\newcommand{\heading}[1]{
  \vspace{0.3cm} \noindent \textbf{#1} \newline
}

\newcommand{\icon}[1]{\tikz[baseline=-3pt]\node[inner sep=0pt,outer sep=0pt]{\includegraphics[height=1.1em]{#1}};}


\title{
  \textbf{ADAMS} \\
  {\Large \textbf{A}dvanced \textbf{D}ata mining \textbf{A}nd \textbf{M}achine
  learning \textbf{S}ystem} \\
  {\Large Module: adams-maps} \\
  \vspace{1cm}
  \includegraphics[width=2cm]{images/maps-module.png} \\
}
\author{
  Peter Reutemann \\
  Dale Fletcher
}

\setcounter{secnumdepth}{3}
\setcounter{tocdepth}{3}

\begin{document}

\begin{titlepage}
\maketitle

\thispagestyle{empty}
\center
\begin{table}[b]
	\begin{tabular}{c l l}
		\parbox[c][2cm]{2cm}{\copyright 2012-2013} &
		\parbox[c][2cm]{5cm}{\includegraphics[width=5cm]{images/coat_of_arms.pdf}}
	\end{tabular}
	\includegraphics[width=12cm]{images/cc.png} \\
\end{table}

\end{titlepage}

\tableofcontents
%\listoffigures
%\listoftables

%%%%%%%%%%%%%%%%%%%%%%%%%%%%%%%%%%%
\chapter{Flow}

The \textit{adams-maps} module adds basic GIS-capability (geographic
information system) to the ADAMS framework. This is done mainly in the
form of GPS support: data types and spreadsheet support.


%%%%%%%%%%%%%%%%%%%%%%%%%%%%%%%%%%%
\chapter{Data}

The following spreadsheet readers are available:
\begin{tight_itemize}
	\item \textit{ArcInfoASCIIGridReader} -- reads files in
	ARC/INFO ASCII GRID format\cite{esrigrid}.
\end{tight_itemize}

\noindent The following spreadsheet object handlers are available:
\begin{tight_itemize}
	\item \textit{GPSDecimalDegrees}
	\item \textit{GPSDecimalMinutes}
	\item \textit{GPSDecimalSeconds}
\end{tight_itemize}

\noindent The following conversions are available:
\begin{tight_itemize}
	\item \textit{SpreadSheetToKML} -- turns a spreadsheet into
	KML\cite{kml}.
\end{tight_itemize}


%%%%%%%%%%%%%%%%%%%%%%%%%%%%%%%%%%%
\chapter{Databases}

The following sections describe how you can utilize the GIS capabilities
of various database systems.

\section{Postgresql}
In order to make use of GIS functionality, you need to install the
\textit{postgis} extension\cite{postgis} for PostgreSQL\cite{postgresql}.

\subsection{Installation}
For any new database, you need to install the extensions first:
\begin{verbatim}
CREATE EXTENSION postgis;
\end{verbatim}

\subsection{Data types}
In order to make use of GPS coordinates in a table, e.g., for calculating
distances between them, you need to create a combined column. The following
example queries create a combined column \textit{lon\_lat} from the columns
\textit{lon} and \textit{lat} of the \textit{some\_table} table.
As data type, SRID 4269\cite{srid4269} (also called NAD 83) is used.

\noindent First, you need to create the column \textit{lon\_lat} and
 define it as data type \textit{POINT}:

\begin{verbatim}
SELECT AddGeometryColumn(
  'public', 'some_table', 'lon_lat', 4269, 'POINT', 2);
\end{verbatim}

\noindent Second, you need to fill it with data, using the \textit{ST\_SetSRID}
function, taking points generated from the \textit{lon} and \textit{lat}
columns:
\begin{verbatim}
UPDATE some_table
SET lon_lat = ST_SetSRID(ST_Point(lon, lat), 4269);
\end{verbatim}

\noindent Having done this, you can execute queries, e.g., retrieving records that have
a distance of less than 50km to the GPS coordinates of longitude of 28.136015
and latitude of -14.613297, ordered by increasing distance. For distance
calculation the \textit{ST\_distance\_sphere} method is used:
\begin{verbatim}
SELECT *, ST_distance_sphere(
  ST_GeomFromText('POINT(28.136015 -14.613297)', 4269),
  ST_GeomFromText(ST_asText(lon_lat), 4269)) AS dist
FROM
  some_table
WHERE ST_distance_sphere(
  ST_GeomFromText('POINT(28.136015 -14.613297)', 4269),
  ST_GeomFromText(ST_asText(lon_lat), 4269)) < 50 * 1000.0
ORDER BY
  dist ASC
\end{verbatim}

\noindent \textbf{NB:} These distance calculations are computationally quite expensive
and, if possible, you should have indices on the \textit{lat}, \textit{long}
and \textit{lon\_lat} columns, as well as restricting the window for the longitude
and latitude that you are performing this query on.

\section{MySQL}
Spatial extensions in MYSQL\footnote{\url{https://dev.mysql.com/doc/refman/5.6/en/spatial-extensions.html}{}}


%%%%%%%%%%%%%%%%%%%%%%%%%%%%%%%%%%%
% This work is made available under the terms of the
% Creative Commons Attribution-ShareAlike 4.0 license,
% http://creativecommons.org/licenses/by-sa/4.0/.

\begin{thebibliography}{999}
	% to make the bibliography appear in the TOC
	\addcontentsline{toc}{chapter}{Bibliography}

    % references
	\bibitem{adams}
		\textit{ADAMS} -- Advanced Data mining and Machine learning System \\
		\url{https://adams.cms.waikato.ac.nz/}{}
		
	\bibitem{poi}
		\textit{Apache POI} -- the Java API for Microsoft Documents \\
		\url{http://poi.apache.org/}{}

	\bibitem{fastexcel}
		\textit{fastexcel} -- Generate and read big Excel files quickly \\
		\url{https://github.com/dhatim/fastexcel}{}

\end{thebibliography}


\end{document}
