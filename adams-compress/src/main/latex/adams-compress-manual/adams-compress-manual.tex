% This work is made available under the terms of the
% Creative Commons Attribution-ShareAlike 4.0 license,
% http://creativecommons.org/licenses/by-sa/4.0/.

\documentclass[a4paper]{book}

\usepackage{wrapfig}
\usepackage{graphicx}
\usepackage{hyperref}
\usepackage{multirow}
\usepackage{scalefnt}
\usepackage{tikz}

% watermark -- for draft stage
%\usepackage[firstpage]{draftwatermark}
%\SetWatermarkLightness{0.9}
%\SetWatermarkScale{5}

\input{latex_extensions}

\title{
  \textbf{ADAMS} \\
  {\Large \textbf{A}dvanced \textbf{D}ata mining \textbf{A}nd \textbf{M}achine
  learning \textbf{S}ystem} \\
  {\Large Module: adams-compress} \\
  \vspace{1cm}
  \includegraphics[width=2cm]{images/compress-module.png} \\
}
\author{
  Peter Reutemann
}

\setcounter{secnumdepth}{3}
\setcounter{tocdepth}{3}

\begin{document}

\begin{titlepage}
\maketitle

\thispagestyle{empty}
\center
\begin{table}[b]
	\begin{tabular}{c l l}
		\parbox[c][2cm]{2cm}{\copyright 2012-2025} &
		\parbox[c][2cm]{5cm}{\includegraphics[width=5cm]{images/coat_of_arms.pdf}} \\
	\end{tabular}
	\includegraphics[width=12cm]{images/cc.png} \\
\end{table}

\end{titlepage}

\tableofcontents
%\listoffigures
%\listoftables

%%%%%%%%%%%%%%%%%%%%%%%%%%%%%%%%%%%
\chapter{Introduction}
Compressing and decompressing files, archiving multiple files, extracting files
from archives. These are all actions that are very common and most of the time
manually performed. Using ADAMS, these steps can be automated thanks to the 
various transformers available that handle archives.

In general, there are two different kinds of actors:
\begin{tight_itemize}
  \item multi-file archives (compress/decompress)
  \item single-file algorithms (compression/decompression)
\end{tight_itemize}
The latter only work on a single file, like for instance gzip. ZIP, on the
other hand, works with multiple files.
The single-file algorithms can compress/decompress byte arrays in addition
to files.

The following chapters cover the actors for compression and decompression in 
more detail.


%%%%%%%%%%%%%%%%%%%%%%%%%%%%%%%%%%%
\chapter{Boolean conditions}
The following boolean flow conditions can be applied to file (String/File objects)
or byte arrays:
\begin{tight_itemize}
  \item IsBzip2Compressed
  \item IsGzipCompressed
  \item IsRarCompressed
  \item IsXzCompressed
  \item IsZipCompressed
  \item IsZstdCompressed
\end{tight_itemize}

%%%%%%%%%%%%%%%%%%%%%%%%%%%%%%%%%%%
\chapter{Compression}
In order to compress one or more files, you can use the following transformers:
\begin{tight_itemize}
	\item \textbf{single files/byte arrays}
	\begin{tight_itemize}
		\item Bzip2 (see \cite{bzip2})
		\item GZIP (see \cite{gzip})
		\item Lzf (see \cite{lzf})
		\item Lzma (see \cite{lzma})
		\item Xz (see \cite{xz})
		\item Zstd (see \cite{zstd})
	\end{tight_itemize}
	\item \textbf{multiple files}
	\begin{tight_itemize}
		\item Tar (see \cite{tar})
		\item ZIP (see \cite{zip})
	\end{tight_itemize}
\end{tight_itemize}

\heading{Single file}
These transformers take a single file as input, which gets 
compressed.\footnote{adams-compress-gzip\_single\_file.flow}
You have the choice of selecting a target file. If not, then the generated
archive gets placed in the same directory as the input file. Optionally,
you can also remove the original file. This is useful if you simply want
to save disk-space and compress all files in a directory, but don't need
the original files anymore.

\heading{Byte array}
Byte arrays can be compressed using these transformers as
well, generating a compressed byte array.\footnote{adams-compress-byte\_array\_handling.flow}

\heading{Multiple files}
Transformers that manage archives instead of only compressing single files,
you need to supply them with an array of file names that should get added
to the archive (no incremental adding possible at the moment). Using the
\textit{stripPath} regular expression, you can exert control over what of
the (most likely absolute) path of the file names should end up in the
archive. In order to strip the complete path from the names, simply use
``.*'' as expression.\footnote{adams-compress-generate\_zip.flow}


%%%%%%%%%%%%%%%%%%%%%%%%%%%%%%%%%%%
\chapter{Decompression}
In order to decompress a compressed file or extract one more files from a
multi-file archive, you have the following transformers available:
\begin{tight_itemize}
	\item \textbf{single file archives/compressed byte arrays}
	\begin{tight_itemize}
		\item UnBzip2 (see \cite{bzip2})
		\item UnGZIP (see \cite{gzip})
		\item UnLzf (see \cite{lzf})
		\item UnLzma (see \cite{lzma})
		\item UnXz (see \cite{xz})
		\item UnZstd (see \cite{zstd})
	\end{tight_itemize}
	\item \textbf{multi-file archives}
	\begin{tight_itemize}
		\item UnRAR (see \cite{rar})
		\item UnTar (see \cite{tar})
		\item UnZIP (see \cite{zip})
	\end{tight_itemize}
\end{tight_itemize}

\heading{Single file archives}
When using transformers for decompressing the content of a single file archive,
you can choose whether you would like to extract the content to a different 
directory and even whether to use a different file name (by default, the file name
is the one with the compression's suffix).\footnote{adams-compress-decompress\_gzipped\_file.flow}

\heading{Compressed byte array}
Compressed byte arrays can be decompressed using these transformers as
well, restoring the original byte array.\footnote{adams-compress-byte\_array\_handling.flow}

\heading{Multi-file archives}
By default, all files from a multi-file archive get extracted. If you only 
require a subset of the files, you can use the \textit{regExp} option to
limit the files extracted to the ones that match the regular expression (you
can also invert the matching sense).\footnote{adams-compress-extract\_from\_zip.flow}
Optionally, you can choose whether to re-create the directory structure 
stored in the archive.


%%%%%%%%%%%%%%%%%%%%%%%%%%%%%%%%%%%
\chapter{Incremental creation}
It is also possible to create multi-file archives in stages using the following
actors:
\begin{tight_itemize}
    \item \texttt{NewArchive} -- initializes the archive and forwards a data structure to be used
    by the following two actors
    \item \texttt{AppendArchive} -- for adding a file or an item from storage to the archive,
    can be used multiple times
    \item \texttt{CloseArchive} -- finalizes and closes the archive
\end{tight_itemize}
The flow \textit{adams-compress-incremental\_zip.flow} demonstrates the use of these actors.

%%%%%%%%%%%%%%%%%%%%%%%%%%%%%%%%%%%
% This work is made available under the terms of the
% Creative Commons Attribution-ShareAlike 4.0 license,
% http://creativecommons.org/licenses/by-sa/4.0/.
%
% Version: $Revision$

\begin{thebibliography}{999}
	% to make the bibliography appear in the TOC
	\addcontentsline{toc}{chapter}{Bibliography}

    % references
	\bibitem{adams}
		\textit{ADAMS} -- Advanced Data mining and Machine learning System \\
		\url{https://adams.cms.waikato.ac.nz/}{}

	\bibitem{jfreechart}
		\textit{JFreeChart} -- is a free 100\% Java chart library that
		makes it easy for developers to display professional quality
		charts in their applications. \\
		\url{http://www.jfree.org/jfreechart/}{}

	\bibitem{xchart}
		\textit{XChart} -- is a light-weight Java library for plotting data. \\
		\url{https://github.com/knowm/XChart}{}

\end{thebibliography}


\end{document}
