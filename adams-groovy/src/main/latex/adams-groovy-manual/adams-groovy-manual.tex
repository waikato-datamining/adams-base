% Copyright (c) 2009-2012 by the University of Waikato, Hamilton, NZ.
% This work is made available under the terms of the 
% Creative Commons Attribution-ShareAlike 4.0 license,
% http://creativecommons.org/licenses/by-sa/4.0/.
%
% Version: $Revision$

\documentclass[a4paper]{book}

\usepackage{wrapfig}
\usepackage{graphicx}
\usepackage{hyperref}
\usepackage{multirow}
\usepackage{scalefnt}
\usepackage{tikz}

% watermark -- for draft stage
%\usepackage[firstpage]{draftwatermark}
%\SetWatermarkLightness{0.9}
%\SetWatermarkScale{5}

\input{latex_extensions}

\title{
  \textbf{ADAMS} \\
  {\Large \textbf{A}dvanced \textbf{D}ata mining \textbf{A}nd \textbf{M}achine
  learning \textbf{S}ystem} \\
  {\Large Module: adams-groovy} \\
  \vspace{1cm}
  \includegraphics[width=4cm]{images/groovy_logo.png} \\
}
\author{
  Peter Reutemann
}

\setcounter{secnumdepth}{3}
\setcounter{tocdepth}{3}

\begin{document}

\begin{titlepage}
\maketitle

\thispagestyle{empty}
\center
\begin{table}[b]
	\begin{tabular}{c l l}
		\parbox[c][2cm]{2cm}{\copyright 2009-2024} &
		\parbox[c][2cm]{5cm}{\includegraphics[width=5cm]{images/coat_of_arms.pdf}} \\
	\end{tabular}
	\includegraphics[width=12cm]{images/cc.png} \\
\end{table}

\end{titlepage}

\tableofcontents
%\listoffigures
%\listoftables

%%%%%%%%%%%%%%%%%%%%%%%%%%%%%%%%%%%
\chapter{Introduction}
Developing a new actor for ADAMS is not hard. But it can take too long, if you
simply want to test something: writing the code, compiling, packaging and,
finally, executing it. Dynamic languages that run in the Java Virtual Machine
(JVM), like Groovy (\cite{groovy}), come in handy. Here, you simply have to
write the code and then execute it. There is no need to compile or package anything.

Groovy's syntax is similar to the Java one. Here is an example, taking a Java
class:
\begin{verbatim}
import java.util.Vector;
public class FunkyVector extends Vector {
  public String toString() {
    String result = "Funky output: ";
    result += super.toString();
    return result;
  }
}
\end{verbatim}
And turning it into a Groovy script:
\begin{verbatim}
import java.util.Vector
class FunkyVector extends Vector {
  public String toString() {
    def result = "Funky output: "
    result += super.toString()
    return result
  }
}
\end{verbatim}
Apart from the missing semi-colons and the missing data type when defining a
local variable, the code looks pretty much the same. But Groovy also comes
with other features, like Closures
\footnote{\url{http://groovy.codehaus.org/Closures}{}}
or lazy transformation
\footnote{\url{http://groovy.codehaus.org/Lazy+transformation}{}}.
For more details, you might want to check out the Groovy documentation
\footnote{\url{http://groovy.codehaus.org/User+Guide}{}}.


%%%%%%%%%%%%%%%%%%%%%%%%%%%%%%%%%%%
\chapter{Writing actors}
Writing a Groovy actor works just like writing a regular ADAMS actor in Java.

\section{Superclass and wrapper}
First, you create an empty text file for you new 
actor\footnote{You can also simply use the \textit{inlineScript} option if you don't 
want to use a separate file.}. Second, you choose what superclass you want to derive it from:
\begin{tight_itemize}
	\item \textbf{Standalone} -- \texttt{adams.flow.standalone.AbstractScript}
	\item \textbf{Source} -- \texttt{adams.flow.source.AbstractScript}
	\item \textbf{Transformer} -- \texttt{adams.flow.transformer.AbstractScript}
	\item \textbf{Sink} -- \texttt{adams.flow.sink.AbstractScript}
\end{tight_itemize}
This also determines, which ADAMS wrapper actor you need to use for executing
your external script:
\begin{tight_itemize}
	\item \textbf{Standalone} -- \texttt{adams.flow.standalone.Groovy}
	\item \textbf{Source} -- \texttt{adams.flow.source.Groovy}
	\item \textbf{Transformer} -- \texttt{adams.flow.transformer.Groovy}
	\item \textbf{Sink} -- \texttt{adams.flow.sink.Groovy}
\end{tight_itemize}
You simply use your script file as the \texttt{scriptFile} property and now you
only have to write the actual code.

\newpage
\section{Implementation}
As for writing your code, you merely have to implement all the abstract methods
from your \texttt{AbstractScript} superclass. The following code shows a
minimalistic \textit{transformer}, which accepts and generates \texttt{Integer}
objects. In the \texttt{doExecute()} method, as it stands, it does not do anything with
the incoming data, it merely forwards a new \texttt{Token} with the data
it received.
\begin{verbatim}
import adams.flow.core.Token
import adams.flow.transformer.AbstractScript

class SimpleTransformer extends AbstractScript {
  public String globalInfo() {
    return "My simple transformer"
  }
  public Class[] accepts() {
    return [Integer.class] as Object[]
  }
  public Class[] generates() {
    return [Integer.class] as Object[]
  }
  protected String doExecute() {
    m_OutputToken = new Token(m_InputToken.getPayload())
    return null
  }
}
\end{verbatim}
A slightly more complex version computes the square of the incoming integer:
\begin{verbatim}
  ...
  protected String doExecute() {
    Integer input = (Integer) m_InputToken.getPayload()
    m_OutputToken = new Token(input * input)
    return null
  }
  ...
\end{verbatim}

\newpage
\section{Parameters}
\label{parameters}
Of course, most of the actors that you will write, will require some form of
parametrization. Instead of defining options in the script itself, the ADAMS
wrapper actor takes on the role of providing parameters. Each of the
\texttt{Groovy} wrapper actors has a property called \texttt{scriptOptions}
which takes a blank-separated list of key-value pairs (``key=value'').

These options are available in the Groovy script via the
\texttt{getAdditionalOptions()} method, returning an
\texttt{adams.flow.core.AdditionalOptions} container object. This container
object offers retrieval of the options via their key:
\begin{tight_itemize}
	\item \texttt{getBoolean(String)} and \texttt{getBoolean(String,Boolean)}
	\item \texttt{getInteger(String)} and \texttt{getInteger(String,Integer)}
	\item \texttt{getDouble(String)} and \texttt{getDouble(String,Double)}
	\item \texttt{getString(String)} and \texttt{getString(String,String)}
\end{tight_itemize}
The second method listed allows you to specify a default value, in case the
option was not supplied.

Assuming that we require an additional option called \texttt{add}, we can use
this parameter to add to our incoming integer value in order to generate output:
{\small
\begin{verbatim}
  ..
  protected String doExecute() {
    Integer input = (Integer) m_InputToken.getPayload()
    m_OutputToken = new Token(new Integer(input + getAdditionalOptions().getInteger("add", 1)))
    return null
  }
  ..
\end{verbatim}
}

%%%%%%%%%%%%%%%%%%%%%%%%%%%%%%%%%%%
\chapter{Writing conversions}
Just like with Groovy actors, Groovy conversions have follow the same principle,
by being implemented as regular conversion schemes in ADAMS.

\section{Superclass and wrapper}
First, you create an empty text file for you new 
conversion\footnote{You can also simply use the \textit{inlineScript} option if you don't 
want to use a separate file.}. Second, you use the following superclass to derive your
Groovy conversion from:
\begin{verbatim}
  adams.data.conversion.AbstractScript
\end{verbatim}
The external script is executed using the following wrapper conversion:
\begin{verbatim}
  adams.data.conversion.Groovy
\end{verbatim}
You simply use your script file as the \texttt{scriptFile} property and now you
only have to write the actual code.

\newpage
\section{Implementation}
As for writing your code, you merely have to implement all the abstract methods
from your \texttt{AbstractScript} superclass. The following code shows a
minimalistic conversion, which accepts and generates \texttt{Double}
objects. In the \texttt{doConvert()} method, it merely divides the incoming
doubles by 100.
\begin{verbatim}
import adams.data.conversion.AbstractScript

class SimpleConversion
  extends AbstractScript {

  public String globalInfo() {
    return "Just divides the incoming numbers by 100."
  }

  public Class accepts() {
    return Double.class
  }

  public Class generates() {
    return Double.class
  }

  protected Object doConvert() throws Exception {
    return m_Input / 100
  }
}
\end{verbatim}
\section{Parameters}
As in how to use parameters, see section \ref{parameters} as it works the
same way as for actors.

%%%%%%%%%%%%%%%%%%%%%%%%%%%%%%%%%%%
\chapter{Troubleshooting}

\begin{tight_itemize}
    \item \texttt{Unexpected character: '"' @ line X, column Y.} -- if you are using an \textbf{inline script}
    via a variable, then having new lines (\texttt{\\n}) in your strings will render the script unusable, due
    to the expansion of the new line characters.
\end{tight_itemize}


%%%%%%%%%%%%%%%%%%%%%%%%%%%%%%%%%%%
% This work is made available under the terms of the
% Creative Commons Attribution-ShareAlike 4.0 license,
% http://creativecommons.org/licenses/by-sa/4.0/.
%
% Version: $Revision$

\begin{thebibliography}{999}
	% to make the bibliography appear in the TOC
	\addcontentsline{toc}{chapter}{Bibliography}

    % references
	\bibitem{adams}
		\textit{ADAMS} -- Advanced Data mining and Machine learning System \\
		\url{https://adams.cms.waikato.ac.nz/}{}

	\bibitem{jfreechart}
		\textit{JFreeChart} -- is a free 100\% Java chart library that
		makes it easy for developers to display professional quality
		charts in their applications. \\
		\url{http://www.jfree.org/jfreechart/}{}

	\bibitem{xchart}
		\textit{XChart} -- is a light-weight Java library for plotting data. \\
		\url{https://github.com/knowm/XChart}{}

\end{thebibliography}


\end{document}
