% This work is made available under the terms of the
% Creative Commons Attribution-ShareAlike 4.0 license,
% http://creativecommons.org/licenses/by-sa/4.0/.

\documentclass[a4paper]{book}

\usepackage{wrapfig}
\usepackage{graphicx}
\usepackage{hyperref}
\usepackage{multirow}
\usepackage{scalefnt}
\usepackage{tikz}
\usepackage{varwidth}

% watermark -- for draft stage
%\usepackage[firstpage]{draftwatermark}
%\SetWatermarkLightness{0.9}
%\SetWatermarkScale{5}

% Version: $Revision$

\newenvironment{tight_itemize}{
\begin{itemize}
  \setlength{\itemsep}{1pt}
  \setlength{\parskip}{0pt}
  \setlength{\parsep}{0pt}}{\end{itemize}
}

\newenvironment{tight_enumerate}{
\begin{enumerate}
  \setlength{\itemsep}{1pt}
  \setlength{\parskip}{0pt}
  \setlength{\parsep}{0pt}}{\end{enumerate}
}

% if you just need a simple heading
% Usage:
%   \heading{the text of the heading}
\newcommand{\heading}[1]{
  \vspace{0.3cm} \noindent \textbf{#1} \newline
}

\newcommand{\icon}[1]{\tikz[baseline=-3pt]\node[inner sep=0pt,outer sep=0pt]{\includegraphics[height=1.1em]{#1}};}


\title{
  \textbf{ADAMS} \\
  {\Large \textbf{A}dvanced \textbf{D}ata mining \textbf{A}nd \textbf{M}achine
  learning \textbf{S}ystem} \\
  {\Large Module: adams-net} \\
  \vspace{1cm}
  \includegraphics[width=2cm]{images/net-module.png} \\
}
\author{
  Peter Reutemann
}

\setcounter{secnumdepth}{3}
\setcounter{tocdepth}{3}

\begin{document}

\begin{titlepage}
\maketitle

\thispagestyle{empty}
\center
\begin{table}[b]
	\begin{tabular}{c l l}
		\parbox[c][2cm]{2cm}{\copyright 2012-2025} &
		\parbox[c][2cm]{5cm}{\includegraphics[width=5cm]{images/coat_of_arms.pdf}} \\
	\end{tabular}
	\includegraphics[width=12cm]{images/cc.png} \\
\end{table}

\end{titlepage}

\tableofcontents
\listoffigures
%\listoftables

%%%%%%%%%
% Email %
%%%%%%%%%

\chapter{Email}
Flows are ideal for being run as background jobs (``-headless'' flag). For
example, importing or processing data in batches can be done at night time. 
Of course, you want to be notified if something went wrong or some predictions
are off. Adding the \textit{SendEmail} sink to existing flows, allows for automatic
sending of emails that were generated by the \textit{CreateEmail} transformer: 
if everything is OK then send an email to the customer,
otherwise send an email to sysadmin. The \textit{CreateEmail} actor adds all incoming
file names, e.g., the array output of a \textit{DirectoryLister} source, as
attachments before sending the email off to the specified recipients. You can
also define a custom subject and body. Variables can be placed in subject and
body alike, as they get expanded when creating the email.

With the built-in IMAP\cite{imap} support, you can also monitor email accounts and
react to messages and/or their attachments.


\section{SMTP}
In order to be able to send emails, ADAMS needs to know what SMTP\cite{smtp} server to
connect to. The following example configures ADAMS to send emails using a Gmail
account\cite{gmail_other_clients}.

There are two ways of configuring SMTP:
\begin{tight_itemize}
	\item \textit{globally} -- using the preferences
	\item \textit{per flow} -- using the \textit{SMTPConnection} standalone actor
\end{tight_itemize}

\subsection{Global SMTP settings}
For configuring email globally, use the dialog available from the main
menu (\textit{Program $\rightarrow$ Preferences $\rightarrow$ Email}), as depicted in Figure
\ref{email_setup}. The placeholders \textit{USER} and \textit{PASSWORD} have to
be replaced with the actual user credentials and \textit{YOUR NAME} with the
actual user's name, of course.

Alternatively, you can also simply create a properties file in the
\texttt{\$HOME/.adams} directory, called \texttt{Email.props}. The content for
the Gmail setup\cite{gmail_app_pw} would look like this:

{\scriptsize
\begin{verbatim}
Enabled=true
SmtpUseSsl=false
SmtpStartTls=true
SmtpProtocols=TLSv1.2
SmtpPassword=YOUR_APP_PASSWORD  # https://support.google.com/accounts/answer/185833?hl=en
SmtpServer=smtp.gmail.com
SmtpTimeout=30000
SmtpPort=587
SmtpRequiresAuthentication=true
SmtpUser=USER
DefaultAddressFrom=YOUR NAME <USER@gmail.com>
DefaultSignature=
\end{verbatim}}

\begin{figure}[htb]
  \centering
  \includegraphics[width=10.0cm]{images/email_setup.png}
  \caption{Email preferences}
  \label{email_setup}
\end{figure}

\subsection{Command-line}
Using the \textit{adams.core.net.SimpleMailer} command-line tool, you can read
and send previously saved emails. The following command-line loads/sends all
files in the directory \textit{/some/where} that end with \textit{.props} using the
\textit{PropertiesEmailFileReader} reader:
\begin{verbatim}
java -cp lib/* adams.core.net.SimpleMailer
  -env adams.env.Environment
  -reader adams.data.io.input.PropertiesEmailFileReader
  -watch-dir /some/where
  -regexp ".*.props"
  -D 1
\end{verbatim}
After a file got sucessfully loaded and sent, it gets (by default) the
\textit{.sent} extension appended. If the process failed, the \textit{.failed}
extension is used (by default).
Use the \textit{-h} or \textit{-help} option on the command-line to list all
the available options.

\clearpage
\section{Flow components}
The following standalone actors are available:
\begin{tight_itemize}
	\item \textit{SMTPConnection} -- for configuring a SMTP server connection.
    \item \textit{IMAPConnection} - defines the IMAP server parameters.
\end{tight_itemize}
The following source actors are available:
\begin{tight_itemize}
    \item \textit{IMAPOperation} - for IMAP operations that only generate output
\end{tight_itemize}
The following transformers are available:
\begin{tight_itemize}
	\item \textit{CreateEmail} -- creates an Email object. Interprets incoming files
	as attachments.
	\item \textit{EmailFileReader} -- reads email files with the specified email reader.
    \item \textit{IMAPOperation} - for operations that take input and generate output
    \item \textit{SaveEmailAttachments} - saves attachments of received emails to the specified directory
\end{tight_itemize}
The following sinks are available:
\begin{tight_itemize}
	\item \textit{EmailFileWriter} -- writes Email objects to files using the specified
	email writer.
	\item \textit{EmailViewer} -- displays an Email object; can be used in conjunction 
	with the \textit{DisplayPanelManager} as well.
	\item \textit{SendEmail} -- sends Email objects to a SMTP server.
\end{tight_itemize}
The following conversions are available:
\begin{tight_itemize}
    \item \textit{ReceivedEmailToMap} - turns the properties of a received email into a
    Java map, to make them easier to access in the flow.
\end{tight_itemize}

\heading{SMTPConnection/IMAPConnection}
If you do not want to store the password with the flow - after all, the password
is only obscured with base64 encoding - you can enable the 
\textit{promptForPassword} option. This will prompt the user with a dialog 
for entering a password to be used for the connection.

For Gmail, you will need to enable app passwords\cite{gmail_app_pw}.


\clearpage
\section{Addressbook}
ADAMS also offers a very simple addressbook for emails. Figure \ref{email_address_book}
shows a screenshot of it.
\begin{figure}[htb]
  \centering
  \includegraphics[width=10.0cm]{images/email_addressbook.png}
  \caption{Email address book}
  \label{email_address_book}
\end{figure}


\section{Support email}
In order to make it easier for users or clients, you can set up a
\textit{support email}, which makes it easy to send through error reports.
These error reports get the \textit{system info} and current content
of the \textit{console window} automatically attached.

To enable this feature, configure the email setup through the preferences
and fill in a \textit{Support email address}. After the application has
been restarted, you get a \textit{Send error report} menu item in the
\textit{Program} menu and the \textit{Send error report} action in the flow's
notification area is enabled as well (this one also attaches the flow that
generated the error for further analysis).


%%%%%%%
% SSH %
%%%%%%%

\chapter{SSH}
ADAMS also comes with SSH support, thanks to the JSch library
\footnote{\url{http://www.jcraft.com/jsch/}{}}. This allows for secure remote
access. At the time of writing, RSA
\footnote{\url{http://en.wikipedia.org/wiki/RSA_\%28algorithm\%29}{}} and DSA
\footnote{\url{http://en.wikipedia.org/wiki/Digital_Signature_Algorithm}{}} are
supported for public key encryption schemes.

The following actors are available:
\begin{tight_itemize}
	\item \textit{SSHConnection} -- standalone actor that defines the server
	connection (user/password or public key authentication).
	\item \textit{SSHExec} -- for executing remote commands.
	\item \textit{ScpFrom} -- for copying a file \textit{from} a remote host using
	secure copy.
	\item \textit{ScpTo} -- for copying a file \textit{to} a remote host using
	secure copy.
	\item \textit{SFTPDelete} -- for delete a remote file via secure FTP.
	\item \textit{SFTPGet} -- for obtaining a remote file via secure FTP.
	\item \textit{SFTPSend} -- for sending a file via secure FTP to a remote host.
\end{tight_itemize}
The \textit{SSHConnection} standalone allows you to prompt the user as runtime
to enter a password, in case you do not want to store the connection's password
with the flow setup. The user has to enter the password through a dialog that
pops up when the flow gets executed.

There are also several \textit{searchlets} available for the
\textit{FileSystemSearch} source actor that allow the listing of remote
files.


%%%%%%%
% SSL %
%%%%%%%

\chapter{SSL}
How ADAMS handles SSL, e.g., for HTTPS connections, can be configured in the
\textit{SSL} preferences. Figure \ref{ssl_setup} shows a screenshot of the
default settings.

\begin{figure}[htb]
  \centering
  \includegraphics[width=10.0cm]{images/ssl_setup.png}
  \caption{SSL preferences}
  \label{ssl_setup}
\end{figure}


%%%%%%%
% SMB %
%%%%%%%

\chapter{SMB}
SMB support is available through the following actors are available:
\begin{tight_itemize}
	\item \textit{SMBConnection} -- standalone actor that defines the server
	connection (domain/user/password).
	\item \textit{SMBGet} -- for obtaining a remote file from a Windows host.
	\item \textit{SMBSend} -- for sending a file to a remote Windows host.
\end{tight_itemize}

There are also several \textit{searchlets} available for the
\textit{FileSystemSearch} source actor that allow the listing of remote
files.


%%%%%%%%%%%%%%%%%%%
% Remote commands %
%%%%%%%%%%%%%%%%%%%

\chapter{Remote commands}

The \textit{adams-net} module adds several connection schemes for the
remote command framework:

\begin{tight_itemize}
	\item \textit{ScpConnection} -- sends commands as files using secure
	copy (\textit{scp}).
	\item \textit{SSHConnection} -- offers SSH tunnelling, useful when
	remote servers are locked down.
\end{tight_itemize}


%%%%%%%%
% Misc %
%%%%%%%%

\chapter{Miscellaneous}
Some other basic, but useful actors are the following:
\begin{tight_itemize}
	\item \textit{Browser} -- opens the system's default browser with the specified
	URL.
	\item \textit{DownloadFile} -- downloads a file via HTTP and saves it to disk.
	\item \textit{DownloadContent} -- downloads (textual) content via HTTP and 
	forwards it.\footnote{adams-net-download\_content.flow}
	\item \textit{Html4Display} -- this sink displays HTML 4 code (and supports clicking on links)
	\item \textit{HttpPostFile} -- the transformer posts a file as multipart/form-data
	to the specified URL and returns the response.
	\item \textit{HttpRequest} -- the source allows accessing URLs (POST/GET, sending data,
	cookies), returns the reponse as text; the transformer allows
	sending the incoming text via HTTP request (and optional headers).
	\item \textit{MimeType} -- determines the mime-type of
	files.\footnote{adams-net-mimetype.flow}
	\item \textit{Socket} -- source and sink allow you to receive and send
	strings and byte arrays via sockets.
	\item \textit{URLSupplier} -- outputs one or more URLs.
	\item \textit{WebSocketClient} -- allows executing of clients that implement
	the websocket protocol\footnote{\url{https://en.wikipedia.org/wiki/WebSocket}{}}.
	\item \textit{WebSocketServer} -- allows running of websocket servers.
\end{tight_itemize}
The following conversions are available:
\begin{tight_itemize}
    \item \textit{Base64ToByteArray} -- decodes a Base64 string as byte array.
    \item \textit{Base64ToString} -- decodes a Base64 string as string.
    \item \textit{ByteArrayToBase64} -- encodes a byte array as Base64 string.
    \item \textit{StringArrayToURLParameters} -- converts a string array of
    key-value pairs into a string to used in a URL.
    \item \textit{StringToBase64} -- encodes a string as Base64.
    \item \textit{StringToURL} -- converts a string into a URL object.
    \item \textit{URLEncode} -- encodes a string into a URL-conform string.
    \item \textit{URLDecode} -- decodes a URL string into a regular string.
    \item \textit{URLParametersToStringArray} -- converts the parameters of
    a URL (after the '?') back into a string array of key-value pairs.
    \item \textit{URLToString} -- converts a URL object into a string.
\end{tight_itemize}


%%%%%%%%%%%%%%%%%%%%%%%%%%%%%%%%%%%
% This work is made available under the terms of the
% Creative Commons Attribution-ShareAlike 4.0 license,
% http://creativecommons.org/licenses/by-sa/4.0/.

\begin{thebibliography}{999}
	% to make the bibliography appear in the TOC
	\addcontentsline{toc}{chapter}{Bibliography}

    % references
	\bibitem{adams}
		\textit{ADAMS} -- Advanced Data mining and Machine learning System \\
		\url{https://adams.cms.waikato.ac.nz/}{}
		
	\bibitem{poi}
		\textit{Apache POI} -- the Java API for Microsoft Documents \\
		\url{http://poi.apache.org/}{}

	\bibitem{fastexcel}
		\textit{fastexcel} -- Generate and read big Excel files quickly \\
		\url{https://github.com/dhatim/fastexcel}{}

\end{thebibliography}


\end{document}
