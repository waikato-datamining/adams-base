% This work is made available under the terms of the
% Creative Commons Attribution-ShareAlike 4.0 license,
% http://creativecommons.org/licenses/by-sa/4.0/.

\documentclass[a4paper]{book}

\usepackage{wrapfig}
\usepackage{graphicx}
\usepackage{hyperref}
\usepackage{multirow}
\usepackage{scalefnt}
\usepackage{tikz}

% watermark -- for draft stage
%\usepackage[firstpage]{draftwatermark}
%\SetWatermarkLightness{0.9}
%\SetWatermarkScale{5}

\input{latex_extensions}

\title{
  \textbf{ADAMS} \\
  {\Large \textbf{A}dvanced \textbf{D}ata mining \textbf{A}nd \textbf{M}achine
  learning \textbf{S}ystem} \\
  {\Large Module: adams-json} \\
  \vspace{1cm}
  \includegraphics[width=2cm]{images/json-module.png} \\
}
\author{
  Peter Reutemann
}

\setcounter{secnumdepth}{3}
\setcounter{tocdepth}{3}

\begin{document}

\begin{titlepage}
\maketitle

\thispagestyle{empty}
\center
\begin{table}[b]
	\begin{tabular}{c l l}
		\parbox[c][2cm]{2cm}{\copyright 2019-2025} &
		\parbox[c][2cm]{5cm}{\includegraphics[width=5cm]{images/coat_of_arms.pdf}} \\
	\end{tabular}
	\includegraphics[width=12cm]{images/cc.png} \\
\end{table}

\end{titlepage}

\tableofcontents
%\listoffigures
%\listoftables

%%%%%%%%%%%%%%%%%%%%%%%%%%%%%%%%%%%
\chapter{Flow}
Flows cannot only be saved as JSON\cite{json} files (and read in again), but flows
themselves can process JSON data structures as well. \\
The following sources are available:
\begin{tight_itemize}
	\item \textit{NewJsonStructure} -- for creating empty JSON data structures
	like objects or arrays.
\end{tight_itemize}
The following transformers are available:
\begin{tight_itemize}
	\item \textit{DeleteJsonValue} -- removes the specified key/value pair
	from the JSON object passing through.
	\item \textit{GetJsonKeys} -- outputs all named elements
	of a JSON object.
	\item \textit{GetJsonValue} -- outputs the named value
	from a JSON object, can use simple key or a JSON
	path\cite{jsonpath}.
	\item \textit{JsonFileReader} -- reads the specific JSON file and forwards
	a JSON object/array.
	\item \textit{SetJsonValue} -- stores a value in a JSON object, can
	use simple key or a JSON path\cite{jsonpath}.
\end{tight_itemize}
The following sinks are available:
\begin{tight_itemize}
	\item \textit{JsonDisplay} -- displays a JSON object in a browseable
	tree structure.
	\item \textit{JsonFileWriter} -- writes the JSON object/array to disk.
\end{tight_itemize}
The following conversion are available:
\begin{tight_itemize}
	\item \textit{ArrayToJsonArray} -- generates a JSON array from any object	array.
	\item \textit{ListToJsonArray} -- generates a JSON array from a list.
	\item \textit{JsonArrayToArray} -- turns a JSON array into a regular Java	object array.
	\item \textit{JsonArrayToList} -- turns a JSON array into a regular Java list.
	\item \textit{JsonObjectToMap} -- turns the JSON object into a simple Map.
	\item \textit{JsonToSpreadSheet} -- turns the JSON object into a spreadsheet (i.e., flattening it).
	\item \textit{JsonToString} -- turns the JSON object/array into a string.
	\item \textit{ListToJson} -- turns a \textit{java.util.List} object	into a JSON object.
	\item \textit{MapToJson} -- turns a \textit{java.util.Map} object	into a JSON object.
	\item \textit{ReportToJson} -- turns an \textit{adams.data.report.Report} objectinto a JSON object.
	\item \textit{SpreadSheetToJson} -- turns a spreadsheet into a JSON array object.
	\item \textit{StringToJson} -- parses the string and generates a JSON object/array.
	\item \textit{StringToObject} -- maps the JSON data onto an object of the specified class.
\end{tight_itemize}

%%%%%%%%%%%%%%%%%%%%%%%%%%%%%%%%%%%
\chapter{Tools}

\section{Pretty print JSON}
In order to preserve space, JSON is often optimized and removes any unnecessary
whitespaces. However, for a human to inspect such data, it is much more useful
to have it properly indented, aka \textit{pretty printed}. Figure \ref{pretty_print_json}
shows a screenshot of the \textit{Pretty print JSON} tool that allows you to
turn JSON into a more human-readable format.
\begin{figure}[htb]
  \centering
  \includegraphics[width=12.0cm]{images/pretty_print_json.png}
  \caption{Pretty print JSON}
  \label{pretty_print_json}
\end{figure}

\section{OptionLister}
The \texttt{adams.core.option.OptionLister} tool allows you to export all the options for
ADAMS classes that implement the \texttt{adams.core.option.OptionHandler} interface
as JSON\footnote{Uses the \texttt{adams.core.option.JsonClassDescriptionProducer} class under the hood}.
The following values are output for each class:
\begin{tight_itemize}
    \item \texttt{option} -- the command-line flag
    \item \texttt{property} -- the Java introspection property
    \item \texttt{type} -- the name of the base class (including primitives)
    \item \texttt{multiple} -- whether the option supports single or multiple values (typically array)
    \item \texttt{help} -- if available, the help string for the option
\end{tight_itemize}
Executes the tool with the \texttt{-h/--help} flag to see all its options.

%%%%%%%%%%%%%%%%%%%%%%%%%%%%%%%%%%%
% This work is made available under the terms of the
% Creative Commons Attribution-ShareAlike 4.0 license,
% http://creativecommons.org/licenses/by-sa/4.0/.
%
% Version: $Revision$

\begin{thebibliography}{999}
	% to make the bibliography appear in the TOC
	\addcontentsline{toc}{chapter}{Bibliography}

    % references
	\bibitem{adams}
		\textit{ADAMS} -- Advanced Data mining and Machine learning System \\
		\url{https://adams.cms.waikato.ac.nz/}{}

	\bibitem{jfreechart}
		\textit{JFreeChart} -- is a free 100\% Java chart library that
		makes it easy for developers to display professional quality
		charts in their applications. \\
		\url{http://www.jfree.org/jfreechart/}{}

	\bibitem{xchart}
		\textit{XChart} -- is a light-weight Java library for plotting data. \\
		\url{https://github.com/knowm/XChart}{}

\end{thebibliography}


\end{document}
